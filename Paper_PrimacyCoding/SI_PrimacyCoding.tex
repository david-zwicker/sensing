\documentclass[twocolumn, superscriptaddress]{revtex4}

\pdfoutput=1

\usepackage{amsmath}
\usepackage{amssymb}
\usepackage{graphicx}
\usepackage{units}

\newcommand{\MI}{I}
\newcommand{\Nc}{{N_{\rm c}}}
\newcommand{\Nl}{{N_{\rm l}}}
\newcommand{\Nr}{{N_{\rm r}}}
\newcommand{\Sbin}{\hat S}

\input{"common"}

\renewcommand{\eqref}[1]{\ref{#1}}

\renewcommand{\figref}[1]{Fig.\nolinebreak[4]\hspace{0.25em}\nolinebreak[4]\ref{#1}}
\renewcommand{\theequation}{\textbf{\arabic{equation}}}


\newlength{\figwidth}
\setlength{\figwidth}{86mm}
\setlength{\figwidth}{\columnwidth}

\renewcommand{\theequation}{S\arabic{equation}}
\renewcommand{\thefigure}{S\arabic{figure}}
\renewcommand{\thesection}{S\arabic{section}}


\begin{document}


\title{Supporting Information: Primacy Coding Theory}

\date{\today}

\author{David Zwicker}
\email{dzwicker@seas.harvard.edu}
\homepage{http://www.david-zwicker.de}
\affiliation{School of Engineering and Applied Sciences, Harvard University, Cambridge, MA 02138, USA}
\affiliation{Kavli Institute for Bionano Science and Technology, Harvard University, Cambridge, MA 02138, USA}

%\author{Michael P. Brenner}
%\email{brenner@seas.harvard.edu}
%\affiliation{School of Engineering and Applied Sciences, Harvard University, Cambridge, MA 02138, USA}
%\affiliation{Kavli Institute for Bionano Science and Technology, Harvard University, Cambridge, MA 02138, USA}

\maketitle

\section{Receptor sensitivities}

\subsection{Equilibrium binding model}

We consider a simple model where receptors~$R_n$ get activated when they bind ligands~$L_i$.
This binding is described by the chemical reaction \mbox{$R_n + L_i \rightleftharpoons R_nL_i$}, where $R_nL_i$ is the receptor-ligands complex.
The equilibrium of the reaction is characterized by a binding constant~$K_{ni}$, which reads
\begin{equation}
	K_{ni} = \exp\left(\frac{E_{ni}}{\kb T}\right)	
	\label{eqn:binding_constant}
	\;,
\end{equation}
where $E_{ni}$ is the interaction energy between receptor~$n$ and ligand~$i$. 
In equilibrium, the concentrations denoted by square brackets obey $[R_nL_i] = K_{ni} \cdot [R_n][L_i]$.
Hence,
\begin{align}
	[R_nL_i] &= \frac{c^{\rm rec}_n K_{ni} c_i}{1 + \sum_i K_{ni} c_i}
	\label{eqn:binding}
	\;,
\end{align}
where we consider the case where multiple ligands compete for the same receptor.
Here, $c^{\rm rec}_n=[R_n] + \sum_i [R_nL_i]$ denotes the fixed concentration of receptors and $c_i = [L_i]$ is the concentration of free ligands.
We consider a simple receptor model in which the excitation~$e^{\rm rec}_n$ of a receptors of type~$n$ is proportional to the concentration of bound ligands,
\begin{align}
	e^{\rm rec}_n &= \alpha_n \sum_i  [R_nL_i]
	\label{eqn:receptor_occupancy}
	\;,
\end{align}
where $\alpha_n$ characterizes the excitability of receptor type~$n$.
As discussed in the main text, the excitations of all receptors of a given type are accumulated in the respective glomeruli, whose excitation~$e^{\rm glo}_n$ is thus given by $e^{\rm glo}_n = N^{\rm rec}_n e^{\rm rec}_n$, where $N^{\rm rec}_n$ is the number of receptors of type~$n$.
In the simple case of binary outputs, a glomerulus becomes active if its excitation exceeds a threshold~$t_n$, $a_n = \Theta(e^{\rm glo}_n - t_n)$, where $\Theta(z)$ denotes the Heaviside step function.
We consider the case $\alpha_n c^{\rm rec}_n \gg t_n$, where the glomerulus signals before the associated receptors become saturated.
In this case, we can linearize \Eqref{eqn:binding} and introduce the rescaled quantities
\begin{align}
	e_n &= \frac{e^{\rm glo}_n}{t_n}
& \text{and} &&
	S_{ni} &= \frac{\alpha_n N^{\rm rec}_n c^{\rm rec}_n}{t_n} \, K_{ni}
	\label{eqn:sensitivities}
\end{align}
to obtain \Eqsref{eqn:excitation}--\eqref{eqn:receptor_activity} of the main text.

A simple theory~\cite{Lancet1993} predicts that the interaction energies~$E_{ni}$ between receptors and ligands are normal distributed.
For the receptor model described above, this implies log-normal distributed binding constant~$K_{ni}$, see \Eqref{eqn:binding_constant}.
In this case, the sensitivities~$S_{ni}$ will also be log-normal distributed, see \Eqref{eqn:sensitivities}.

\subsection{Measured receptor sensitivities}
Response matrices have been measured experimentally for flies~\cite{Muench2015} and humans~\cite{Mainland2015}.
The fly database has been constructed by merging data from many studies that used various methods to measure receptor responses~\cite{Muench2015}.
It contains a non-zero response for 5482 receptor-ligand pairs, covering all 52 receptors that are present in flies.
\figref{fig:sensitivities}A in the main text shows the histogram of the logarithm of the associated sensitivities together with a normal distribution with the same mean and variance as the data.

The only comprehensive study of human olfactory receptors used a luciferase assay to measure receptor responses \textit{in vitro}~\cite{Mainland2015}.
They report the intensity of clones of 511 human olfactory receptors in response to various concentrations of 73 ligands.
Typically, the intensity of a given receptor-ligand pair is monotonously increasing as a function of ligand concentration~$c$.
We normalize the intensity to lie between $0$ and $1$ and fit a hyperbolic tangent function to determine the concentration~$c_*$ at which the normalized intensity reaches $0.5$.
Here, the only fit parameters are the concentration~$c_*$  and the slope of the tangent function at this point.
We exclude poor fits, where the relative error in either parameters is above $\unit[50]{\%}$.
This leaves us with 203 of the 623 receptor-ligand combinations, for which we then define the sensitivity as $c_*^{-1}$.
\figref{fig:sensitivities}B in the main text shows the histogram of the logarithm of these sensitivities together with a normal distribution with the same mean and variance as the data.

\section{Order statistics}
The order statistics of the excitations~$\vect e$ are denotes by $e_{(1)} < e_{(2)} < \ldots < e_{(\Nr)}$.
In the simple case where the excitations~$e_n$ are independent and identically distributed, the probability distribution function of the order statistics reads
\begin{align}
	f_{E_{(n)}}(e) =
		\frac{N_{\rm r}![F_E(e)]^{n-1}[1-F_E(e)]^{N_{\rm r} - n}}{(n-1)!(N_{\rm r}-n)!} f_E(e)
	\;,
\end{align}
where $f_E(e)$ and $F_E(e)$ are the probability and cumulative distribution function of the the excitations~$e_n$.
The expectation value can be approximated as~\cite[Eq.~(4.5.1)]{David1970}
\begin{align}
	\mean{e_{(n)}} = F_E^{-1}\left( \frac{n}{\Nr + 1} \right)
	\label{eqn:en_order_statistics_mean}
	\;.
\end{align}
The average excitation threshold~$\mean\eta$ can then also be expressed as $\mean\eta = \mean{e_{(n^*)}}$.
Here $n^*$ must be chosen such that on average $\Nc$ receptors are activated.
This is the case if $\Nc = \Nr P(e_n > \mean\eta) = \Nr[1 - F_E(\mean\eta)]$, which yields
$n^* = \Nr - \Nc + 1 - \Nc\Nr^{-1}$.
The excitation threshold thus reads
\begin{align}
	\mean\eta \approx F_E^{-1}\left( \frac{\Nr - \Nc}{\Nr} \right)
	\label{eqn:excitation_threshold}
	\;.
\end{align}

\section{Receptor response}

\subsection{Discriminability of two ligands}
We next consider how well two ligands can be discriminated.
For simplicity, we consider the case where all entries of the sensitivity matrix are independently drawn from the same distribution.

In the simple case where each ligand is presented individually, the respective activation patterns~$\vect a$ are uncorrelated and independent of the ligand concentrations.
The expected Hamming distance~$h$ between them is
\begin{align}
	\mean{h} &= 2\Nc\left(1 - \frac{\Nc}{\Nr}\right)
	\label{eqn:discriminability_independent}
	\;.
\end{align}
For large $\Nr$, this comes close to the maximal $h$ of $2\Nc$.

We next ask how the activation pattern is changed when a ligand at concentration~$c_1$ is added to another ligand at concentration~$c_1$.
During this process, the excitation threshold changes from $\mean\eta_0$ given by \Eqref{eqn:excitation_threshold} to $\mean\eta_\rho = (1 + \rho)\mean\eta_0$, where $\rho=c_2/c_1$ is the concentration ratio.
We quantify the expected change of the activation pattern by the Hamming distance~$h$, which can be calculated from the probability that a given receptor changes its state.
Here, either the receptor was inactive and becomes active when the second ligand is added.
Alternatively, the receptor could have been active and gets inactivated by the increased excitation threshold.
We call the two respective probabilities~$p_{\rm on}$ and $p_{\rm off}$, such that
\begin{align}
	\mean h &= \Nr \cdot(p_{\rm on} + p_{\rm off})
	\label{eqn:mixture_hamming}
	\;.
\end{align}
We first calculate $p_{\rm on}$, for which the receptor must be inactive if only the first ligand is present, \ie its excitation must be smaller than the threshold~$\rho_0$.
The probability distribution of finding such receptors is given by~$f_E(e^{(1)})$ for $0 \le e^{(1)} <\rho_0$.
Given a value of $e^{(1)}$, the probability that the receptor gets activated by adding the second ligand reads $P(e^{(1)} + \rho e^{(2)} > \eta_\rho | e^{(1)})$, where $e^{(2)}$ is distributed as $F_E$.
Integrating over all possible~$e^{(1)}$, we get
\begin{align}
	p_{\rm on} &= \int_0^{\eta_0}
		\! P\left(e^{(1)} + \rho e^{(2)} > \eta_\rho \,\middle | \, e^{(1)} \right) f_E\bigl(e^{(1)}\bigr) \, \diff e^{(1)}
\notag\\
	&= \int_0^{\eta_0}
	\left[1 - F_E\left(\frac{\eta_\rho - e^{(1)}}{\rho}\right)\right]  f_E\bigl(e^{(1)}\bigr) \, \diff e^{(1)}
	\;.
\end{align}
Similarly, we find
\begin{align}
	p_{\rm off} &=\int_{\eta_0}^{\infty}
		F_E\left(\frac{\eta_\rho - e^{(1)}}{\rho}\right)  f_E\bigl(e^{(1)}\bigr) \, \diff e^{(1)}
	\label{eqn:mixture_poff}
	\;.
\end{align}
Combining \Eqsref{eqn:mixture_hamming}--\eqref{eqn:mixture_poff}, we can thus calculate the expected change~$h$ of the activation pattern if a second ligand is added to a mixture.
Taking the limit $\lim_{\rho \rightarrow \infty} h$ we recover \Eqref{eqn:discriminability_independent}, which is expected since in the case $c_2 \gg c_1$ the the first ligand should not contribute and the activation patterns should independent.


\todo{Scaling for small~$\rho$. Expect: $h \propto \Nc\rho$, $p \propto \rho\Nc/\Nr$}
For small~$\rho$, we have
\begin{align}
	F_E\left(\frac{\eta_\rho - e^{(1)}}{\rho}\right) = \Theta(\eta_0 - y) + \order{\rho^2}
\end{align}
Consequently, $p_{\rm on} = \frac\Nc\Nr F_E(\eta_0)$ and $p_{\rm off} = \frac\Nc\Nr[1 -  F_E(\eta_0)]$, such that $h \propto \Nc$.

\todo{Consider mixtures with overlap}
Finally, we also consider two mixtures with overlap

\bibliographystyle{unsrt}
\bibliography{bibdesk}


\end{document}
