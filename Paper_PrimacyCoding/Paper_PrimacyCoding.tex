\documentclass[twocolumn, superscriptaddress]{revtex4}

\pdfoutput=1

\usepackage{amsmath}
\usepackage{amssymb}
\usepackage{graphicx}
\usepackage{units}

\newcommand{\MI}{I}
\newcommand{\Nc}{{N_{\rm c}}}
\newcommand{\Nl}{{N_{\rm l}}}
\newcommand{\Nr}{{N_{\rm r}}}
\newcommand{\Sbin}{\hat S}

\input{"common"}

\renewcommand{\eqref}[1]{\ref{#1}}

\renewcommand{\figref}[1]{Fig.\nolinebreak[4]\hspace{0.25em}\nolinebreak[4]\ref{#1}}
\renewcommand{\theequation}{\textbf{\arabic{equation}}}

\begin{document}

\title{Primacy Coding Theory}
\date{\today}

\author{David Zwicker}
\email{dzwicker@seas.harvard.edu}
\homepage{http://www.david-zwicker.de}
\affiliation{School of Engineering and Applied Sciences, Harvard University, Cambridge, MA 02138, USA}
\affiliation{Kavli Institute for Bionano Science and Technology, Harvard University, Cambridge, MA 02138, USA}

%\author{Michael P. Brenner}
%\email{brenner@seas.harvard.edu}
%\affiliation{School of Engineering and Applied Sciences, Harvard University, Cambridge, MA 02138, USA}
%\affiliation{Kavli Institute for Bionano Science and Technology, Harvard University, Cambridge, MA 02138, USA}

\begin{abstract}
Abstract\end{abstract}

\keywords{Olfaction | Sensing | Information Theory}

\maketitle

\newlength{\figwidth}
\setlength{\figwidth}{86mm}
\setlength{\figwidth}{\columnwidth}

 \section{Results}
 \subsection{Physical Model}
 The odor~$\vect c = (c_1, c_2, \ldots, c_{\Nl})$ excites the receptors, which is described by the excitation~$\vect e = (e_1, e_2, \ldots, e_{\Nr})$.
 For simplicity, we consider the linear model
 \begin{align}
	e_n &= \sum_{i=1}^{\Nl} S_{ni} c_i
	\;,
\end{align}
where the sensitivity matrix~$S_{ni}$ denotes how sensitive receptor~$n$ is to ligand~$i$.
In primacy coding, the $\Nc$ most excited receptor will get active and contribute to the signal.
Consequently, only receptors who's excitation exceeds a threshold~$\eta$ contribute to the activity vector~$\vect a = (a_1, a_2, \ldots, a_{\Nr})$,
\begin{align}
	a_n &= \begin{cases}
		0 & e_n \le \eta \\
		1 & e_n > \eta \;,
	\end{cases}
\end{align}
where $\eta$ must be chosen such that $\sum_n a_n = \Nc$.

The threshold~$\eta$ can be determined for each realization of~$\vect e$ by ordering the excitations, $e_{(1)} < e_{(2)} < \ldots < e_{(\Nr)}$, where the index in the bracket denotes the position in the sorted list.
The threshold can then be defined as $\eta = e_{(\Nr - \Nc)}$, such that the $\Nc$ receptors associated with $e_{(\Nr - \Nc + 1)}, \ldots, e_{(\Nr)}$ get activated.
 

\begin{acknowledgments}
This research was funded by the German Science Foundation through ZW 222/1-1.
\end{acknowledgments}

\bibliographystyle{unsrt}
\bibliography{bibdesk}

\end{document}
